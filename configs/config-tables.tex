% Подключаем нужные пакеты для таблиц и оформления
\usepackage{booktabs}   % Для красивых горизонтальных линий (\toprule, \midrule, \bottomrule)
\usepackage{multirow}   % Для объединения ячеек по вертикали
\usepackage{caption}    % Для \ContinuedFloat (продолжение таблицы)

% ------------------------
% Шаблон: Многоуровневый заголовок
% ------------------------
\newcommand{\ExampleMultiLevelTable}{%
\begin{table}[h!]
\caption{Пример построения таблицы}
\centering
\begin{tabular}{|c|c|c|c|}
\hline
\multirow{2}{*}{\textbf{Заголовок}} & \multicolumn{2}{c|}{\textbf{Заголовок колонки}} & \multirow{2}{*}{\textbf{Заголовок колонки}}\\
\cline{2-3}
& \textbf{Подзаголовок} & \textbf{Подзаголовок} & \\
\hline
A & B & C & D \\
\hline
\end{tabular}
\end{table}
}

% Шаблон: "Разбитая" таблица (две части под одним номером)
\newcommand{\ExampleSplitTable}{%
  % --- Первая часть ---
  \begin{table}[h!]
    \LeftCaption
    \caption{Пример переноса таблицы}
    \centering
    \begin{tabular}{|c|c|c|c|}
    \hline
    \textbf{Заголовок} & \textbf{Заголовок колонки} & \textbf{Заголовок колонки} & \textbf{Заголовок колонки} \\
    \hline
    1 & 2 & 3 & 4 \\
    \hline
    2 & 3 & 4 & 5 \\
    \hline
    \end{tabular}
  \end{table}

  % --- Продолжение ---
  \begin{table}[h!]
    \ContinuedFloat
    \RightCaption
    \caption*{Продолжение таблицы 5.1}
    \centering
    \begin{tabular}{|c|c|c|c|}
    \hline
    \textbf{Заголовок} & \textbf{Заголовок колонки} & \textbf{Заголовок колонки} & \textbf{Заголовок колонки} \\
    \hline
    3 & 4 & 5 & 6 \\
    \hline
    4 & 5 & 6 & 7 \\
    \hline
    \end{tabular}
  \end{table}
}

\usepackage{caption}
\usepackage{longtable}
\usepackage{etoolbox}
\DeclareCaptionLabelSeparator{emdash}{\space---\space}  % пробел, длинное тире, пробел

% Настройка выравнивания заголовков
\captionsetup[table]{position=t, singlelinecheck=false, justification=raggedright, labelsep=emdash, name=Таблица}
\renewcommand{\thetable}{\thesection.\arabic{table}}

% Переключатель для смены выравнивания заголовков
\newcommand*\RightCaption{\captionsetup{justification=raggedleft}}
\newcommand*\LeftCaption{\captionsetup{justification=raggedright}}

